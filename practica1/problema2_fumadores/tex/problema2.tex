\documentclass[11pt]{article}

%%% PAQUETES

% Idioma
\usepackage[utf8]{inputenc}
\usepackage[spanish]{babel}
\setlength{\parindent}{0pt}

% Matemáticas
\usepackage{amssymb, amsthm}

% Fuentes
\usepackage[T1]{fontenc}
\usepackage[sfdefault, scaled=.85]{roboto}
\usepackage[scaled=.8]{FiraMono}
\usepackage[cmintegrals]{newtxsf}
\usepackage[italic]{mathastext}
\usepackage{textcomp}
\usepackage{wasysym}

% Tablas
\usepackage{multirow}
\usepackage{colortbl}

% Gráficos y colores

\usepackage[x11names, rgb, html]{xcolor}
\usepackage{graphics}
\usepackage{caption}
\usepackage{float}
\usepackage{adjustbox}

% Ajustes del documento
\usepackage{geometry}
\geometry{left=3cm,right=3cm,top=3cm,bottom=3cm,headheight=1cm,headsep=0.5cm}
\usepackage{enumitem}
\usepackage{anysize}

% Entornos personalizados.
\usepackage{mdframed}

% Código
\usepackage{listingsutf8}

%%% COLORES

% Material Design

\definecolor{50}{HTML}{E0F7FA}
\definecolor{300}{HTML}{4DD0E1}
\definecolor{500}{HTML}{00BCD4}
\definecolor{700}{HTML}{0097A7}
\definecolor{900}{HTML}{006064}

% Solarized

\definecolor{sbase03}{HTML}{002B36}
\definecolor{sbase02}{HTML}{073642}
\definecolor{sbase01}{HTML}{586E75}
\definecolor{sbase00}{HTML}{657B83}
\definecolor{sbase0}{HTML}{839496}
\definecolor{sbase1}{HTML}{93A1A1}
\definecolor{sbase2}{HTML}{EEE8D5}
\definecolor{sbase3}{HTML}{FDF6E3}
\definecolor{syellow}{HTML}{B58900}
\definecolor{sorange}{HTML}{CB4B16}
\definecolor{sred}{HTML}{DC322F}
\definecolor{smagenta}{HTML}{D33682}
\definecolor{sviolet}{HTML}{6C71C4}
\definecolor{sblue}{HTML}{268BD2}
\definecolor{scyan}{HTML}{2AA198}
\definecolor{sgreen}{HTML}{859900}

% Colores del documento

\definecolor{text}{RGB}{78,78,78}
\definecolor{accent}{RGB}{129, 26, 24}

%%% LISTINGS

% Listing -> Código fuente
\renewcommand{\lstlistingname}{Código fuente}

% Ajustes para que funcionen bien las tildes de los comentarios

\lstset{
  inputencoding=utf8/latin1
}

% Ajustes de Listings para el documento

\lstset{
  frame=leftline,
  rulecolor=\color{300},
  framerule=2pt,
  % Números de línea
  numbers=left,
  % Margen adicional para alinear los entornos con el resto de párrafos
  xleftmargin=0.7em,
  % Espacio adicional debajo del título
  belowcaptionskip=1\baselineskip,
  % Colores
  basicstyle=\footnotesize\ttfamily\color{sbase00},
  keywordstyle=\color{700},
  commentstyle=\color{300},
  stringstyle=\color{500},
  numberstyle=\color{500},
  % Separar líenas largas en varias líneas
  breaklines=true,
  showstringspaces=false,
  tabsize=2,
}

%%% ENTORNOS PERSONALIZADOS

\newtheoremstyle{ejercicio-style} % Nombre del estilo.
{-0.5em}                          % Espacio por encima.
{}                                % Espacio por debajo.
{\normalfont}                     % Fuente del cuerpo.
{}                                % Identación.
{\bf\sffamily}                    % Fuente para la cabecera.
{.}                               % Puntuación tras la cabecera.
{.5em}                            % Espacio tras la cabecera.
{\thmname{#1}\thmnumber{ #2}\thmnote{ (#3)}}     % Especificación de la cabecera (actual: nombre en negrita).

% Nuevo estilo para definiciones
\newtheoremstyle{ejercicio-style}  % Nombre del estilo
{-0.5em}                                  % Espacio por encima
{}                                  % Espacio por debajo
{\normalfont}                                  % Fuente del cuerpo
{}                                  % Identación
{\bf\sffamily}                      % Fuente para la cabecera
{.}                                 % Puntuación tras la cabecera
{.5em}                              % Espacio tras la cabecera
{\thmname{#1}\thmnote{ #3}}  % Especificación de la cabecera


\mdfdefinestyle{ejercicio-frame}{
  linewidth=2pt, %
  linecolor= 300, %
  backgroundcolor= 50,
  topline=false, %
  bottomline=false, %
  rightline=false,%
  leftmargin=0pt, %
  innerleftmargin=1em, %
  innerrightmargin=1em,
  rightmargin=0pt, %
  innertopmargin=1em,%
  innerbottommargin=1em, %
  splittopskip=\topskip, %
}%

\surroundwithmdframed[style=ejercicio-frame]{ejer}

\theoremstyle{ejercicio-style}
\newtheorem{ejer}{Problema}

\begin{document}

% Cabecera del documento

\begin{tabular*}{\textwidth}{@{\extracolsep{\fill}}!{\color{300}{\vrule width 2pt}}>{\columncolor{50}}clc}
    \noalign{\global\arrayrulewidth=2pt}
    %\arrayrulecolor{300}\hline
    & & \\
    & \Large{Sistemas Concurrentes y Distribuidos} & \\
               & \large{Práctica 1} & \\
               & \large{Problema 2.} & \\
          & \large{} & \\
          & & \\
          & \textsf{Estudiante: Miguel Lentisco Ballesteros } & \\
          & \textsf{Grupo de prácticas:} A1 & \\
          %& \textsf{Fecha de entrega:} TBA & \\
         \multirow{-10}{*}{ \begin{tabular}{c}
        \small{3º curso / 1º cuatr.} \\ Grado Ing. Inform. \\ Doble Grado Ing. \\ Inform. y Mat.
\end{tabular}}  & & \\
    %\hline
\end{tabular*}

\vspace{1cm}


\section*{Los 3 fumadores y el estanquero}
\label{sec::prod_cons}

\begin{ejer} Se nos pide dar una solución al problema del Productor/Consumidor siguiendo la plantilla proporcionada. Para la solución se usará como estructura para el buffer tanto como una estructura $LIFO$ (pila), como $FIFO$ (cola).
\end{ejer}

\subsection*{LIFO}
\label{sec::lifo}
Con el buffer como una pila, para saber donde estamos escribiendo y leyendo tendremos que usar una variable global que funcione como índice, esta variable será incrementada cuando el Productor escriba en el buffer y será decrementada y luego se lee el dato por el Consumidor, representando así el índice la posición donde tiene que escribir o leer respectivamente. \\

Por supuesto tenemos que tener en cuenta que pasa cuando vayan a escribir y leer en el buffer el Consumidor y el Productor van a tener que modificar la variable indice pudiendo probar problemas de escritura/lectura; por ello cuando vayamos a usar esta variable tendremos que usar exclusión mutua: o bien un cerrojo o que sea una variable del tipo $atomic<int>$, siendo más fácil de usar la versión con $atomic$.

\subsection*{FIFO}
\label{sec::lifo}
Si el buffer es una cola, tenemos dos posibilidades en este caso:
\begin{itemize}
	\item En este caso, el bucle del productor es hasta la cantidad de items a producir, luego podemos aprovechar esta ventaja y usar la variable $i$ como el índice ya que simplemente se va a escribir en $i \% buffer\_tam$ cuando el semáforo de escritura lo permita (explicado a continuación), e independientemente del productor; el consumidor irá leyendo en el mismo orden $i \% buffer\_tam$ cuando el semáforo de lectura deje hacerlo, por tanto tendremos variables locales que no interfieren entre sí, no necesitamos exclusión mutua.

	\item Por otro lado, si tuviéramos que llevar la cuenta inicialmente o en cualquier momento saber donde se está escribiendo o leyendo entonces tendríamos que tener dos índices, uno de escritura y otro de lectura, variables globales para saberlo. Igualmente que en el primer caso, estas variables serían independientes entre sí y no habría problemas de carrera, luego de nuevo, no necesitamos exclusión mutua. Se hace igual, sustituyendo $i$ por el índice lectura/escritura correspondiente y incrementando cada índice al escribir/leer en el buffer.
\end{itemize}

\subsection*{Semáforo}
\label{sec::semaforo}
Ahora, para resolver el problema con semáforos, visto ya en clase, solo tendremos que poner dos semáforos, uno que sea para el productor (para escribir) y otro para el consumidor (para leer). Inicializamos el de escritura al tamaño del buffer (puede escribir hasta el máximo) y el de lectura a 0 (tiene que esperar a que el productor escriba), entonces cada vez que se vaya a producir haya un $sem\_wait$ de escritura, y cuando escriba se produzca un $sem\_signal$ de lectura para el consumidor que está esperando en $sem\_wait$ de lectura, pueda leer el dato y dar un $sem\_signal$ para indicar que hay un hueco vacío al productor. \\

El semáforo de escritura sirve para parar al consumidor para que no siga escribiendo si el buffer está completo, y el de lectura para el consumidor que se espere a que haya datos que leer. Obviamente al productor le damos un margen de libertad del tamaño del buffer al principio y al de lectura ninguno, ya que se tiene que esperar a que el productor escriba.

\subsection*{Solución}
\label{sec::solucion}
En los archivos fuentes, se encuentran las soluciones con $FIFO$ y $LIFO$ a este problema siguiendo la plantilla proporcionada y completando debidamente las funciones de consumidor y productor, añadiendo las variables globales consideradas (índices y buffer) y mostrando por pantalla cada vez que se escribe/lee en buffer así como cuando cada hebra acaba su cometido con un $fin$. Por supuesto, está comprobado con muchas veces que funciona correctamente.

\subsection*{Solución LIFO}
Veamos la solución $LIFO$:
\lstinputlisting[language=C++, linerange={13-22, 87-126}, caption=prodcons-lifo.cpp]{../src/prodcons-lifo.cpp}

\subsection*{Solución FIFO}
Veamos la solución $FIFO$:
\lstinputlisting[language=C++, linerange={12-22, 87-124}, caption=prodcons-fifo.cpp]{../src/prodcons-fifo.cpp}
\end{document}
